\documentclass[11pt, a4paper]{book}
\input{package.tex}
\usepackage{fancyhdr}    % Package pour personnaliser les en-têtes et pieds de page

\pagestyle{fancy}        % Utilisation du style "fancy"
\fancyhf{}               % Réinitialisation des en-têtes et pieds de page
\fancyhead[L]{Edoardo Basilico et Vincent Namy }   % Pied de page à gauche
\fancyhead[C]{OC Robotique}  % Pied de page C
\fancyhead[R]{2024-2025 -- Collège Sismondi}  % Pied de page à droite
\fancyfoot[C]{}       % Pied de page au centre avec la date actuelle


\makeatletter% since there's an at-sign (@) in the command name
\let\ps@plain\ps@fancy 
\makeatother


\begin{document}

\setcounter{chapter}{0}
\chapter{Versionnage de fichiers - Tutoriel git}

\section{Github}
\begin{enumerate}
    \item Aller sur github.com et y créer un compte avec son adresse eduge.ch
    \item Se rendre sur le repository \href{https://github.com/edu-ge/Tutoriel-git}{github.com/edu-ge/Tutoriel-git}
    \item Dans le nouveau repo : Clone $\rightarrow$ SSH $\rightarrow$ add a new public key (cf \ref{ssh})
\end{enumerate}

\begin{figure}[h!]
\centering
    \includegraphics[width=0.4\linewidth]{ssh.png}
    \caption{Clone ssh}
    \label{ssh}
\end{figure}

\section{Authentification SSH}

\begin{enumerate}
    \item Sur votre ordinateur, ouvrir un terminal (Konsole)
    \item Entrer \texttt{ssh-keygen -t rsa} afin de générer une nouvelle paire de clés RSA, appuyez sur Entrée pour passer les questions optionnelles.
    
    \item Puis \texttt{gedit .ssh/id\_rsa.pub} et copier l'entièreté de ce fichier dans le champ \texttt{Key} de la page \texttt{Add new SSH Key} ouverte sur votre navigateur, et valider. Cette étape authorise votre clé SSH à accéder à votre compte Github.
    \item Finalement,\texttt{gedit .ssh/config} et collez-y le texte suivant afin d'indiquer à git quel clé utiliser : \begin{lstlisting}[numbers=none]
  Host github.com
      User git
      IdentityFile ~/.ssh/id_rsa
\end{lstlisting}
\end{enumerate}

\section{Git - Clonage}

\begin{enumerate}
    \item Dans le terminal, rendez-vous dans le dossier où vous voulez cloner le projet en utilisant les commandes :
        \subitem \texttt{ls} (lister les dossiers) 
        \subitem et \texttt{cd nom\_dossier} (entrer dans le dossier).
    \item Une fois dans le bon dossier, \texttt{git clone git\@github.com:edu-ge/Tutoriel-git.git} pour télécharger le contenu du projet. Vous devriez voir apparaître un nouveau dossier contenant 3 fichiers.

\end{enumerate}

\section{Git - Utilisation}
\subsection{Mode d'emploi}
Dans un terminal, après s'être rendu dans le dossier du repo : 
\begin{itemize}
    \item \texttt{git pull} : pour télécharger les dernières modifications
    \item \texttt{git status} : pour lister l'état de votre dossier
    \item \texttt{git diff} : pour voir les modifications que vous avez faites localement
    \item \texttt{git commit -a -m "[Protocole réseau] Fix erreur dans l'encryption"} (en remplaçant par un message qui décrit vos modifications.) : pour ajouter vos modifications locales à un commit 
    \item \texttt{git push} : pour pousser vos commits sur le serveur
\end{itemize}


\subsection{1\textsuperscript{er} exercice}
Dans le dossier \texttt{Tutoriel git} que vous venez de cloner:
\begin{enumerate}
    \item Modifier le fichier \texttt{liste\_classe.txt} pour ajouter vos initiales.
    \item Faire un commit.
    \item Ouvrir le fichier \texttt{devine\_mon\_nombre.py} et corriger 1 erreur.
    \item Faire un commit, puis un push.
\end{enumerate}


\section{Git - Nouveau repository - 2 \textsuperscript{ème} exercice}
\begin{enumerate}
    \item Créer un nouveau repository sur GitHub :  Repositories $\rightarrow$ Create new repository $\rightarrow$ Entrer un nom, cocher public et readme et valider
    \item Cloner ce repo sur votre ordinateur
    \item Essayer de modifier le ficher Readme.md et de pousser vos modifs.
    \item Pour travailler à plusieurs sur le même repo,  sur GitHub, dans \texttt{Settings}, cliquer sur \texttt{Add people} et entrer le username à inviter.
    \item Invitez votre voisin·e dans votre projet essayez chacun·e de faire des modifications sur le projet de l'autre.
\end{enumerate}




\end{document}