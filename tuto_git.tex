\documentclass[11pt, a4paper]{book}
\input{package.tex}
\usepackage{fancyhdr}    % Package pour personnaliser les en-têtes et pieds de page

\pagestyle{fancy}        % Utilisation du style "fancy"
\fancyhf{}               % Réinitialisation des en-têtes et pieds de page
\fancyhead[L]{Edoardo Basilico et Vincent Namy }   % Pied de page à gauche
\fancyhead[C]{OC Robotique}  % Pied de page C
\fancyhead[R]{2024-2025 -- Collège Sismondi}  % Pied de page à droite
\fancyfoot[C]{}       % Pied de page au centre avec la date actuelle


\makeatletter% since there's an at-sign (@) in the command name
\let\ps@plain\ps@fancy 
\makeatother


\begin{document}

\setcounter{chapter}{0}
\chapter{Versionnage de fichiers - Tutoriel git}

\section{Github}
\begin{enumerate}
    \item Aller sur github.com et y créer un compte avec son adresse eduge.ch
    \item  Repositories $\rightarrow$ Create new repository $\rightarrow$ Entrer un nom, cocher public et readme et valider
    \item Dans le nouveau repo : Clone $\rightarrow$ SSH $\rightarrow$ add a new public key (cf \ref{ssh})
\end{enumerate}

\begin{figure}[h!]
\centering
    \includegraphics[width=0.5\linewidth]{ssh.png}
    \caption{Clone ssh}
    \label{ssh}
\end{figure}

\section{Authentification SSH}

\begin{enumerate}
    \item Sur votre ordinateur, ouvrir un terminal (Konsole)
    \item Entrer \texttt{ssh-keygen -t rsa} afin de générer une nouvelle paire de clés RSA, appuyez sur Entrée pour passer les questions optionnelles.
    
    \item Puis \texttt{gedit .ssh/id\_rsa.pub} et copier l'entièreté de ce fichier dans le champ \texttt{Key} de la page \texttt{Add new SSH Key} ouverte sur votre navigateur, et valider. Cette étape authorise votre clé SSH à accéder à votre compte Github.
    \item Finalement,\texttt{gedit .ssh/config} et collez-y le texte suivant afin d'indiquer à git quel clé utiliser : \begin{lstlisting}[numbers=none]
  Host github.com
      User git
      IdentityFile ~/.ssh/id_rsa
\end{lstlisting}
\end{enumerate}

\section{Git - Clonage}

\begin{enumerate}
    \item Sur GitHub, retourner à la page du nouveau repository et rouvrir la fenêtre Clone SSH, pour copier le lien SSH du projet. 
    \item Dans le terminal, rendez-vous dans le dossier où vous voulez cloner le projet en utilisant les commandes :
        \subitem \texttt{ls} (lister les dossiers) 
        \subitem et \texttt{cd nom\_dossier} (entrer dans le dossier).
    \item Une fois dans le bon dossier, \texttt{git clone git\@github.com:blablabla} (en collant votre adresse à la place de celle-ci) pour télécharger le contenu du projet. Vous devriez voir apparaître un nouveau dossier contenant le fichier Readme.md.

\end{enumerate}

\section{Git - Utilisation}
Dans un terminal, après s'être rendu dans le dossier du repo : 
\begin{enumerate}
    \item Pour télécharger les dernières modifications, entrez \texttt{git pull}
    \item Pour lister l'état de votre dossier, entrez \texttt{git status}
    \item Pour voir les modifications que vous avez faites localement,  \texttt{git diff}
    \item Pour ajouter vos modifications locales à un commit,  \texttt{git commit -a -m "[Protocole réseau] Fix erreur dans l'encryption"} (en remplaçant par un message qui décrit vos modifs.)
    \item Pour pousser vos commits sur le serveur :   \texttt{git push}

Essayer de modifier le ficher Readme.md et de pousser vos modifs.
\end{enumerate}

\section{Git - Partage}
Pour travailler à plusieurs sur le même repo,  sur GitHub, dans \texttt{Settings}, cliquer sur \texttt{Add people} et entrer le username à inviter.

Invitez votre voisin·e dans votre projet essayez chacun·e de faire des modifications sur le projet de l'autre.

\end{document}